\chapter{Experimental Results}\label{ch:6}
We implement our optimizations in Linux/KVM version 3.0, which has paravirtual extensions for Power. To measure performance of {\em unmodified} KVM and KVM with our optimizations, we disabled the paravirtual extensions. We perform our experiments on Freescale QorIQ P2020 platform, which is optimized for single-threaded performance per watt for networking and telecom applications. Our system has a 1.2GHz processor with 32KB L1 cache and 512 KB L2 cache. We use RAMdisk for our experiments to eliminate I/O overheads.

Our benchmarks are described in Table~\ref{tab:kvm_performance}. We use four macrobenchmarks, namely Linux boot, Echo spawn, Find and Lame. These benchmarks have also been used in a previous performance study\cite{kvm_ppc_exittimings}. The others are microbenchmarks from the widely-used {\tt lmbench} toolset\cite{lmbench} and {\tt unixbench} toolset \cite{unixbench}. Linux boot and Echo spawn execute a large number of privileged instructions, while Find executes relatively fewer privileged instructions. Lame has to do some IO and a lot of computation that does not need to exit to the hypervisor, thus representing an average compute-intensive workload. 

{\tt Lmbench} is a performance analysis suite of micro-benchmarks for UNIX/POSIX. The microbenchmarks are designed to provide the latency and bandwidth deatils of diferent system operations. The microbenchmarks are extensible in nature and target specific subsytems and operations to measure hardware performances for example fork benchmark targets the performance anlaysis of process creation only.

{\tt Unixbench} is also a benchmarking suite to analyze the system performances of Unix-like systems. It also contains the microbenchmarks to test various aspects of system performance. These micro-benchmarks are also specific in nature and targets different subsytems of the system. A major difference with the lmbench benchmarking suite is that while lmbench suite provides the latency of the different subsystems it can't be used to compare the tlb misses, privileged exits and tracing page faults since the number of executions of the micro-benchmark depends on the latency of the benchmark itself. The Unixbench microbenchmarks can be tunes to execute the benchmarks for a fixed number of times giving more coorect data about the overheads and performance optimizations due to adaptive binary translation and tracing.

The number of privileged instructions executed by each benchmark are reported in Table~\ref{tab:vm_exit_stats}. We do not report performance comparisons on purely user-level compute-intensive workloads because they exhibit near bare-metal performance for all cases.

\section{Comparison of native with various virtualization solutions}
We implement the following optimizations in Linux/KVM:
\begin{enumerate}
  \item In-place Binary Translation and Read/Write Tracing
  \item Adaptive Page Resizing
  \item Adaptive Data Mirroring and Translation of Faulting Instructions
\end{enumerate}
Table~\ref{tab:detailed_results} summarizes our performance results before and after these optimizations. Different workloads show different improvements. The improvement primarily depends on a three-way tradeoff between the number of VM exits due to privileged instructions, the number of tracing page faults, and the number of page faults due to increased TLB pressure (TLB misses). Table~\ref{tab:vm_exit_stats} shows the number of VM exits and page faults due to all these three reasons for each workload, before and after our optimizations. We also show the number of exits in the paravirtual solution ({\tt KVM-PV}) for comparison. The number of exits in {\tt KVM-PV} can be considered as a lower-bound on the number of exits achievable by a host-side binary translation solution.

Just using Optimization 1 does not improve performance for a Linux guest. In fact, we find that read/write tracing severely impairs performance because the guest uses a huge 256MB page to map the kernel's code and data. If we constrain the host to only use 4KB shadow pages, the performance improves but is still unacceptably low due to high TLB pressure.

\begin{table*}
\centering
      \begin{tabular}{|l| r r r |} \hline
        Benchmark\verb, ,& {\tt KVM} \verb, , & {\tt Adapt-PR} \verb, , & {\tt Adapt-DM}  \\ \hline
     & \multicolumn{3}{c|}{ Running Time in $sec$}\\ \cline {2-4}  
Linux boot	&	30.03	&	13.02	&	12.39	\\
Echo Spawn	&	21.34	&	7.80	&	6.85	\\
Find	&	1.89	&	0.98	&	0.83	\\
Lame	&	0.56	&	0.51	&	0.50	\\
\cline{2-4}
	   
     & \multicolumn{3}{c|}{Latency in $msec$}\\  \cline{2-4}
syscall	&	0.020	&	0.009	&	0.003	\\
stat	&	0.033	&	0.012	&	0.007	\\
fstat	&	0.021	&	0.009	&	0.004	\\
open/close:	&	0.067	&	0.026	&	0.023	\\
sig hndl	&	0.024	&	0.009	&	0.004	\\
pipe 	&	0.066	&	0.062	&	0.041	\\
fork	&	6.641	&	2.020	&	1.679	\\
exec	&	20.543	&	7.457	&	6.681	\\
sh	&	45.164	&	16.567	&	14.719	\\
 \hline
\cline{1-4}
\multicolumn{1}{|c|}{Unixbench Benchmark}     & \multicolumn{3}{c|}{Number of Calls}\\  \cline{1-4}
Dhrystone 2	&	48110141	&	48904947	&	48957180\\
System Call	&	124854	&	286057	&	652829\\
Pipe-based context switching	&	60746	&	159256	&	240653\\
Process Creation	&	2470	&	7679	&	8714\\
Pipe Throughput	&	188208	&	469538	&	923218\\
Hanoi	&	8689401	&	8690019	&	8836219\\
\cline{1-4}
\multicolumn{1}{|c|}{Unixbench Filesystem Benchmark}     & \multicolumn{3}{c|}{Throughput in KBps with 256 bufsize and 2000 max blocks
}\\  \cline{1-4}
File Read	&	11524	&	24151	&	49830\\
File Write	&	10500	&	19050	&	38200\\
File Copy	&	5432	&	10408	&	20070\\	\cline{1-4}

      \end{tabular}
\caption{\label{tab:detailed_results}Performance Improvements obtained by Adaptive Page Resizing ({\tt Adapt-PR}) and Adaptive Data Mirroring ({\tt Adapt-DM}) optimizations.}
\end{table*} 

\begin{center}
\begin{table*}
\centering
\begin{tabular}{
  |p{1.2cm}@{}|@{ }c@{ }|@{}p{0.9cm}|@{}p{1.2cm}|@{}p{1.0cm}@{}|@{ }c@{ }|@{}p{0.8cm}|@{}p{1.2cm}|@{}p{1.0cm}|@{ }c@{ }|@{}p{0.8cm}|@{}p{1.2cm}|@{}p{0.9cm}|@{ }c@{ }|@{}p{0.8cm}|@{}p{1.2cm}|@{}p{0.9cm}@{}|} \hline
	         &  & \multicolumn{3}{c|}{\tt KVM}& & \multicolumn{3}{c|}{\tt Adapt-PR} & & \multicolumn{3}{c|}{\tt Adapt-DM} & & \multicolumn{3}{c|}{\tt KVM-PV} \\ \hline

           &&{\tt Priv. Exits}&{\tt TLB miss}&{\tt Acc. Viol} &&{\tt Priv. Exits}&{\tt TLB miss}&{\tt Acc. Viol.}&&{\tt Priv. Exits}&{\tt TLB miss}&{\tt Acc. Viol}&&{\tt Priv. Exits}&{\tt TLB miss}&{\tt Acc. Viol} \\ \hline  
Boot	&&	11509859	&	402274	&	13823	&&	48560	&	326825	&	460001	&&	38367	&	345035	&	77467	&&	40451	&	295672	&	14181	\\	\hline
Find	&&	802777	&	2485	&	46	&&	965	&	2655	&	98145	&&	851	&	2782	&	4701	&&	719	&	2392	&	46	\\	\hline
Lame	&&	60312	&	1932	&	23	&&	636	&	2085	&	3126	&&	644	&	2114	&	46	&&	625	&	1916	&	24	\\	\hline
Echo Spawn	&&	9829128	&	419191	&	18005	&&	35823	&	492129	&	518255	&&	35014	&	506295	&	18624	&&	34602	&	427351	&	17011	\\	\hline
Pipe 	&&	28714485	&	979	&	41	&&	18762	&	1113	&	6010468	&&	9185	&	1165	&	75	&&	7648	&	968	&	42	\\	\hline
Hanoi	&&	4005414	&	914	&	40	&&	225211	&	1088	&	151308	&&	224970	&	1169	&	876	&&	224660	&	943	&	41	\\	\hline
Syscall	&&	22541901	&	953	&	41	&&	15105	&	1130	&	6008149	&&	6324	&	1159	&	40	&&	5197	&	948	&	41	\\	\hline
Process Creation	&&	19841882	&	621658	&	60036	&&	100796	&	835408	&	897302	&&	99081	&	846053	&	60026	&&	98569	&	652042	&	60034	\\	\hline
Context switching	&&	26459239	&	1102	&	49	&&	616684	&	7702	&	3609690	&&	610986	&	5567	&	49	&&	608549	&	1059	&	49	\\	\hline
Dhrystone 2	&&	154460	&	973	&	40	&&	7656	&	1127	&	6132	&&	7648	&	1140	&	38	&&	7667	&	1001	&	40	\\	\hline
      \end{tabular}
\caption{\label{tab:vm_exit_stats}Frequency of VM exits due to privileged instructions ({\tt Priv. Exits}), TLB misses ({\tt TLB miss}), and access violations ({\tt Acc. Viol}). The
exits due to access violations are due to read/write accesses to pages which do not have read/write permissions. These exits occur primarily due to read/write tracing.}
\end{table*} 
\end{center}


Optimization~2 adaptively resizes the shadow TLB entries to reduce the number of tracing page faults and results in performance improvements over unmodified KVM. Since we only want to measure the overhead due to adaptive page resizing, translation cache in Optimization~2 has been allocated by the guest using the paravirtual extensions. This ensures that the overhead are only due to the resizing algorithm and not due to the translation cache stealing and tracing page faults on it. The second column ({\tt Adapt-PR}) in Table~\ref{tab:detailed_results} shows the performance of KVM with Optimizations~1~and~2. We first discuss the {\tt lmbench} microbenchmarks. The running time of all microbenchmarks improves with Optimizations~1~and~2. The average improvement is 148\%, with the maximum improvement of 223\% in fork. The improvement in a benchmark primarily depends on the reduction in the number of privileged exits. Different microbenchmarks execute different number of privileged instructions and show different corresponding reductions. We next discuss the improvement on macrobenchmarks using Optimizations~1~and~2. We observe a performance improvement of 101\% on the macrobenchmarks, with the maximum improvement seen in Echo Spawn, which also shows the maximum reduction (99\%) in exits due to privileged instructions. For {\tt unixbench} microbenchmarks, the average improvemnt is XX\%, with maximum improvement in XX. 

For both microbenchmarks and macrobenchmarks, we notice an increase in the number of TLB misses and access violations (due to tracing page faults) on average. The net effect however remains positive. Our next optimization (Optimization~3) further reduces these access violations. We also find that Optimization~3 reduces privileged instruction exits. A slight increase in tlb misses can be observed after Optimization~3 this is due to the fact that in the solution obtained from Optimization~1~and~2  the translation cache is being allocated by the guest while with Optimization~3 steals it from the data section. Stealing space for thetranslation cache results in an increase in the tlb misses since the data section needs to be fragmented in order to trace the stolen space. Also since the translation cache is being allocated by the guest using paravirtual extensions in the guest in solution obtained from Optimization~1~and~2, the scan behavior of the guest while dynamically modifying it's guest is not visible while with Optimization~3 this behvior causes the page resizing algorithm to kick in and results in an increase in tlb misses to save tracing page faults.
This increase has been subdued due to an indirect effect, without Optimization~3 certain guest pages are broken into smaller pages and some of these smaller pages are removed to reduce tracing page faults. With Optimization~3, the number of tracing page faults decreases. This allows Optimization~2 to not have to kick in, allowing pages to remain unbroken and privileged instructions to remain patched. This subdues the TLB misses and also reduces the number of privileged instruction exits. 

Optimization~3 reduces the number of tracing page faults by mirroring the data residing on a traced page and patching the corresponding access instruction to access the data from the mirrored location in future. The reduction is observed in all our benchmarks. The average reduction in the number of access-violation page faults (which are mostly due to tracing page faults) is XX\%.

Overall Optimizations~1,~2~and~3 provide performance comparable to paravirtualization. On our macrobenchmarks, we observe an average speedup of XX\% over unmodified KVM. Optimization~3 provides significant improvements on many benchmarks, including the {\tt find} macrobenchmark, where the improvement is around XX\%. Table~\ref{tab:vm_exit_stats} confirms that these improvements are due to the reduction in exits caused by tracing page faults, and privileged-instruction executions.

\section{Static TLB configured Guest vs Adaptive Page Resizing}
\label{adaptive_pr_results}
We next measure the overhead and effectiveness of our adaptive page resizing algorithm. We statically configured the shadow TLB to the best observed configuration for Linux boot benchmark. In this configuration, we loaded the shadow TLB with 13 entries: one entry of size 4MB ({\tt 0xc0000000-0xc03fffff}), four entries of size 1MB each (covering {\tt 0xc0400000-0xc07fffff}), two entries of size 4MB each (covering {\tt 0xc0800000-0xc0ffffff}), three entries of size 16MB each (covering {\tt 0xc1000000-0xc3ffffff}), and three entries of size 64MB each (covering {\tt 0xc4000000-0xcfffffff}). Of these 13 entries, the first two entries were marked execute-only (kernel code is nearly 5MB long) and the rest remained unmodified. We compared the performance of this configuration with our adaptive page resizing algorithm. We disabled Optimization~3 ({\tt Adapt-DM}) for this experiment. Table~\ref{tab:adaptive_tlb_resizing} summarizes our results. For Linux boot benchmark, we observe that our resizing algorithm performs within 1-2\% of the optimal and within 10-20\% of all the other microbenchmarks as well as macrobenchmarks.

The adaptive-pr solution soon approaches the static configuration as it starts patching the privilege instructions. The tracing page faults we observe in Table~\ref{tab:adaptive_tlb_resizing} are mainly due to the accesses to function dispatch tables and exception handlers. As explained in Section~\ref{adaptive_dm} the adaptive page resizing algorithm keeps on trying to break the shadow tlb entries in order to reduce the trace page faults but has to revert back since the privileged code section accessing these are co-existing with them and are equally frequent. this leads to a slight increase in the tlb misses. 

A more fitting benchmark would be when some data section which is heavily accessed got included in a traced region. {\tt Custom Pipe} and {\tt Custom Syscall} are custom experiments where all the privileged code belonging accessing the exception handlers and function dispatch tables has been removed statically from the guest and {\tt Pipe} \& {\tt Syscall} from unixbench are run on it. Thus the memory region containing the function dispatch tables and exception handlers now act as nearly pure data section. We can observe a 30x decrease in number of tracing page faults in both these custom experiments. An increase in tlb misses can be observed also due to fragmentation but the increase in terms of absolute tlb misses is not so significant as compared to the decrease in exits due to tracing page faults. The same behavior is reflected in their raw scores for these custom experiments.

\begin{table*}
\centering
      \begin{tabular}{|p{2.0cm}| r r |} \hline
        Benchmark\verb, ,& {\tt Static-TLB} \verb, ,& {\tt Adapt-PR} \verb, , \\ \hline

     & \multicolumn{2}{c|}{Running Time in $sec$}\\ \cline {2-3}
Linux boot	&	12.98	&	13.02\\
Echo Spawn	&	8.0	&	7.80\\
Find	&	0.90	&	0.98\\
Lame	&	0.51	&	0.51\\
\cline{1-3}
\multicolumn{1}{|c|}{Lmbench Benchmark}     & \multicolumn{2}{c|}{Latency in $msec$}\\  \cline{1-3}
syscall	&	0.007	&	0.009\\
stat	 &	0.011	&	0.012\\
fstat	&	0.008	&	0.009\\
open/close:	&	0.022	&	0.026\\
sig hndl	&	0.008	&	0.009\\
pipe 	&	0.053	&	0.062\\
fork	&	2.068	&	2.020\\
exec	&	7.792	&	7.457\\
sh	&	17.076	&	16.565\\
\cline{1-3}
\multicolumn{1}{|c|}{Unixbench Benchmark}     & \multicolumn{2}{c|}{Number of Calls}\\  \cline{1-3}
Dhrystone 2	&	48833854	&	48904947\\
System Call	&	344877	&	286057\\
Pipe-based context switching	&	160575	&	159256\\
Process Creation	&	7101	&	7679\\
Pipe Throughput	&	557045	&	469538\\
Hanoi	&	8835343	&	8690019\\
Custom Pipe	&	557045	&	994777\\
Custom Syscall	&	344877	&	739343\\
\cline{1-3}
\multicolumn{1}{|c|}{Unixbench Filesystem Benchmark}     & \multicolumn{2}{c|}{Throughput in KBps with 256 bufsize and 2000 max blocks
}\\  \cline{1-3}
File Read	&	28346	&	24151\\
File Write	&	22450	&	19050\\
File Copy	&	12201	&	10408\\
 \hline
      \end{tabular}
\caption{\label{tab:adaptive_tlb_resizing}Measuring the overhead and performance of adaptive page resizing ({\tt Adapt-PR}) algorithm, in comparison to
a statically configured TLB ({\tt Static-TLB}). The static TLB configuration was setup such that it provides the best performance for the Linux boot benchmark. We disable {\tt Adapt-DM} optimization for this experiment.}
\end{table*} 

\begin{table*}
\centering
\begin{tabular}{|l|@{ }c@{ }|p{1cm}|c|p{1cm}|@{ }c@{ }|p{1cm}|c|p{1cm}|} \hline
	         &&  \multicolumn{3}{c|} {Static TLB Configuration}&& \multicolumn{3}{c|} {Adapt-PR} \\ \hline

           &&{\tt Priv. Exits}&{\tt TLB miss}&{\tt Acc. Viol}&&{\tt Priv. Exits}&{\tt TLB miss}&{\tt Acc. Viol}\\ \hline  
Boot	&&	41770	&	320941	&	452482	&&	48560	&	326825	&	460001	\\	\hline
Find	&&	908	&	2676	&	98113	&&	965	&	2655	&	98145	\\	\hline
Lame	&&	637	&	2052	&	3124	&&	636	&	2085	&	3126	\\	\hline
Echo Spawn	&&	35929	&	485870	&	513294	&&	35823	&	492129	&	518255	\\	\hline
Pipe 	&&	15742	&	1144	&	6008821	&&	18762	&	1113	&	6010468	\\	\hline
Hanoi	&&	224929	&	1116	&	151104	&&	225211	&	1088	&	151308	\\	\hline
Syscall	&&	12791	&	1135	&	6006963	&&	15105	&	1130	&	6008149	\\	\hline
Process Creation	&&	101427	&	815074	&	897920	&&	100796	&	835408	&	897302	\\	\hline
Context switching	&&	614104	&	1260	&	3608524	&&	616684	&	7702	&	3609690	\\	\hline
Dhrystone 2	&&	7663	&	1197	&	6158	&&	7656	&	1127	&	6132	\\	\hline
Custom Pipe	&&	15742	&	1144	&	6008821	&&	15832	&	63963	&	235367	\\	\hline
Custom Syscall	&&	12791	&	1135	&	6006963	&&	12921	&	33050	&	194482	\\	\hline
      \end{tabular}
\caption{\label{tab:adaptive_tlb_resizing_exits}Sources of VM exits, while comparing {\tt Static-TLB} with {\tt Adapt-PR}. We disable {\tt Adapt-DM} optimization for this experiment.}
\end{table*}


